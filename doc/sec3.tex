\documentclass[main]{subfiles} 
\begin{document}
\section{Compiling, Assembling, and Linking}
%\label{sec:}
%%%%%%%%%%%%%%%%%%%%%%%%%%%%%%%%%%%%%%%%%%%%%%%%%%%%%%%%%%%%%%%%%%%%%%%%%%%%%%%%
\subsection{Usual work flow}
preprocess, compile, assemble, link
\subsection{Preprocess}
Replace \texttt{include} with the included files and macros with their defs.

Roughly speaking, nothing happens in this step.
\subsection{Compile and Assemble}
Translate human-readable languages such as Fortran/C/C++ into machine readable
languages, or machine languages.

Usually, these two steps are done by the compilers at one go. So, frequently we
no longer distinguish them and just call them collectively as compiling.

Each .c/.cpp file is compiled into one and only one .o file in this step.

However, we still cannot run the resulting files. The resulted files are not
executables.
\subsection{Link}
Link .o files, add "starting sections", make an executable file.

\subsection{Libraries, Shared Objects}
[static library vs. dynamic library]

[Windows dll, Linux so, MaxOSX dylib]

[good libraries, bad libraries]

\subsection{matlab, Mathematica, python, shell, ruby, R, etc.}
Interpretive, translate on the fly, easily slow by a factor of 20-100 compared
to compiled programs.
%%%%%%%%%%%%%%%%%%%%%%%%%%%%%%%%%%%%%%%%%%%%%%%%%%%%%%%%%%%%%%%%%%%%%%%%%%%%%%%% 
\end{document}

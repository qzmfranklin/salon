\documentclass[main]{subfiles} 
\begin{document}
\section{High Level Languages}
%\label{sec:}
%%%%%%%%%%%%%%%%%%%%%%%%%%%%%%%%%%%%%%%%%%%%%%%%%%%%%%%%%%%%%%%%%%%%%%%%%%%%%%%%
\subsection{A Taste of C}
[hello world example]
\subsection{Compiling and Compilers}
[Compilation of each source file generates one object file]

[Optimizing compiling is artificial intelligence]

[Compiling usually involves at least three phases: parsing, compiling and 
assembling]

[Show the compiler theory book]

[Design of languages]

[Philosophy of C] 
\subsection{Linking and Linker}
[Linking of one or more object file(s) generates one
executable, which is indeed machine instructions]

\subsection{Libraries, Shared Objects}
[static library vs. dynamic library]

[Windows dll, Linux so, MaxOSX dylib]

[good libraries, bad libraries]
\subsection{Fortran}
[ASM helloworld exmaple]

[After compilation, object files are same with C]

[Philosophy of Fortran] 
\subsection{Ex ASM: Fortran \& C}
[linking-cfortran/addtwo.exe]

\subsection{C++ vs. C: Huge Difference}
[name mangling: theory and practice]

[mixed linkage: extern "C"]
\subsection{Ex ASM: C \& C++}
[linking-ccpp/addtwo.exe]

\subsection{Java}
[java virtual machine (JVM)]

[java runtime environment (JRE)]

[java compiler javac generates java virtual machine instructions]

\subsection{Ex Java}
[java/helloworld.exe]

\subsection{Python, Matlab, Mathematica, R}
[Interpretative vs. Compiling]

[Interpreter vs. Compiler]

[Rule of Thumb of Interpretative Tools]
%%%%%%%%%%%%%%%%%%%%%%%%%%%%%%%%%%%%%%%%%%%%%%%%%%%%%%%%%%%%%%%%%%%%%%%%%%%%%%%% 
\end{document}

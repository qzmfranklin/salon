\documentclass[main]{subfiles} 
\begin{document}
\section{What's in a CPU?}
%\label{sec:}
%%%%%%%%%%%%%%%%%%%%%%%%%%%%%%%%%%%%%%%%%%%%%%%%%%%%%%%%%%%%%%%%%%%%%%%%%%%%%%%%
\subsection{1+1=2 Revisited}
\subsection{Operator and Operands}
[graph]
\subsection{Compoud Arithmetic Expressions}
[1+3*5]
\subsection{Tree Structure of Compound Arithmetic Expressions}
[graph:1+3*5]
\subsection{Parsing/Flattening Trees of Arithmetic Expressions}
[lecture:1+3*5] 

\subsection{Machine Representation of Numbers}
[int, long int, unsigned int]

[float, double, long double]

[32bit, 64bit]

[IEEE standard]
\subsection{Instructions and Registers}
[instructions: add, mul, div]

[registers: graph from internet]

[registers may be wider than the data]

[registers may have special uses]

[instruction sets]
\subsection{Vendors and Instruction Sets}
[MMX,SEE,SEE2,SEE3,SEE4,SEE4.1,SEE4.2,AVX,AVX2]

[3DNow!]

[MIPS,MPIS64]

[PPC]

\subsection{Machine Language}
=instruction

[machine dependent]
\subsection{Assembly Language and Assemblers} 
[one-one map between instruction and human readable words]

[machne dependent, but also sometimes maybe software/OS/convention dependent]
\subsection{Ex ASM: 1+3*5}
[cpu/cpdexp.exe]

\subsection{Registers and Caches}
[Registers' schematic plot from wiki]

[Cache size exmaples]

[RAM, disk, speed difference]
%%%%%%%%%%%%%%%%%%%%%%%%%%%%%%%%%%%%%%%%%%%%%%%%%%%%%%%%%%%%%%%%%%%%%%%%%%%%%%%% 
\end{document}

\documentclass[main]{subfiles} 
\begin{document}
\section{What's in a CPU?}
%\label{sec:}
%%%%%%%%%%%%%%%%%%%%%%%%%%%%%%%%%%%%%%%%%%%%%%%%%%%%%%%%%%%%%%%%%%%%%%%%%%%%%%%%
\subsection{1+1=2 Revisited}
What is really going on when we do a \texttt{1+1=2} calculation?

\subsection{Operator and Operands}
[graph]
\subsection{Compound Arithmetic Expressions}
[1+3*5]
precedence, associativity
\subsection{Tree Structure of Compound Arithmetic Expressions}
[graph:1+3*5]
\subsection{Parsing/Flattening Trees of Arithmetic Expressions}
[lecture:1+3*5] 
\subsection{Example}
\texttt{cpu/allis.exe}:
\lstinputlisting{../src/cpu/allis.cpp}

excerpt from \texttt{cpu/allis.s}:
\begin{lstlisting}[language=Ant]
# -- Begin  _Z4funxPKi
...
# parameter 1: %rdi
### 	return n[0] + n[1] * n[2];
        movl      4(%rdi), %eax                                 #11.16
        imull     8(%rdi), %eax                                 #11.23
        addl      (%rdi), %eax                                  #11.23
        ret                                                     #11.23
...
# -- End  _Z4funxPKi
\end{lstlisting}

A complete/near-complete exposition of the above assembly listing would lead to
the discussion of many aspects on the CPU. We will explain each and every of
them in the following sections.

\subsection{Instructions Set Architecture (ISA)}
instruction, opcode : \texttt{0010 0111 1101 0111}

machine language = the set of opcodes

assembly code: \texttt{addl opd1, opd2}

Take the \texttt{x86\_64} architecture as an example:

int classification:
\texttt{x86\_64} instruction sets: see \\
\url{http://en.wikipedia.org/wiki/X86_instruction_listings} \\

[in lecture] proper reference to \url{http://x86.renejeschke.de/} \\


arithmetic(integral,floating point): add, sub, mul, imul

logic,flow control: cmp, jmp(goto)

load/store,memory,cache,tlb,etc.: mov, wb


register classification:

\texttt{x86\_64} registers: see \\
\url{https://www.tortall.net/projects/yasm/manual/html/arch-x86-registers.html}\\
\url{https://software.intel.com/en-us/articles/introduction-to-x64-assembly}\\
\url{http://en.wikipedia.org/wiki/Advanced_Vector_Extensions}

general purpose (arithmetic operand, address container)

special/misc (interrupt/signal, mode, flags, etc.) 

\subsection{Machine Representation of Numbers}
Integral numbers are easy.

Floating number are not that easy.

The IEEE754 standard says: \\
\url{http://steve.hollasch.net/cgindex/coding/ieeefloat.html}

The floating point operations: add, sub, mul, div, fsqrt

The design and implementation ensures that the resulting numbers are the best
approximation to an exact intermediary result. The largest possible relative 
error is $2^{-52}=2.2\times10^{-16}$. Therefore, the number of significant
digits is 16 for double precision numbers.


\subsection{Vendors and Instruction Sets}
[MMX,SEE,SEE2,SEE3,SEE4,SEE4.1,SEE4.2,AVX,AVX2]

[3DNow!]

[PPC]

[ARM]

[Loongson] (!!)

[MIPS,MIPS64]

%%%%%%%%%%%%%%%%%%%%%%%%%%%%%%%%%%%%%%%%%%%%%%%%%%%%%%%%%%%%%%%%%%%%%%%%%%%%%%%% 
\end{document}
